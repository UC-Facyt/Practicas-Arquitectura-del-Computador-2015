\documentclass{article}

\usepackage[spanish]{babel}
\usepackage[utf8]{inputenc}
\usepackage[T1]{fontenc}
\usepackage{graphicx}
\usepackage{wrapfig}
\usepackage{hyperref}
\usepackage{courier}
\usepackage{listings}
\usepackage{xcolor}
 % % % % % % % %
 
\definecolor{mygreen}{rgb}{0,0.6,0}
\definecolor{mygray}{rgb}{0.5,0.5,0.5} 
\definecolor{mymauve}{rgb}{0.58,0,0.82}

% %DEFINIENDO ESTILO PARA SOURCE CODE LISTING PARA ASM 8051 % %
\lstdefinelanguage{ASM8051}{
  keywords={MOV, JMP, DEFSEG, CPL, RETI, SETB, SUBB, ORG, DB, MOVC, CLR, RET, JNB, CALL, CJNE, INC, DEC, END, PUSH, POP, JZ, JB},
  keywordstyle=\color{blue}\bfseries,
  ndkeywords={DELSE, VECTOR, RST, FINC, INEXT, IELSE1, IELSE2, FDCN, DNEXT, DELSE1, DELSE2, FWAIT, WELSE, SIT1, WAIT, AUX, INICIO, EXIT},
  ndkeywordstyle=\color{darkgray}\bfseries,
  identifierstyle=\color{black},
  sensitive=false,
  comment=[l]{;},
  morecomment=[s]{/*}{*/},
  commentstyle=\color{purple}\ttfamily,
  stringstyle=\color{red}\ttfamily,
  morestring=[b]',
  morestring=[b]"
}

\lstset{
   language=ASM8051,
   backgroundcolor=\color{lightgray},
   extendedchars=true,
   basicstyle=\footnotesize\ttfamily,
   showstringspaces=false,
   showspaces=false,
   numbers=left,
   numberstyle=\footnotesize,
   numbersep=9pt,
   tabsize=2,
   breaklines=true,
   showtabs=false,
   captionpos=b
}


\usepackage{anysize}
\marginsize{2.54cm}{2.54cm}{2.54cm}{2.54cm}

\usepackage{setspace}
\onehalfspacing
%En caso de que LaTeX separe las palabras con - de manera incorrecta, usar
%\hyphenation{deci-sión,e-xa-men, otras palabras....}

% %PORTADA
\begin{document}

\begin{centering}
\hrule 	\vspace{0.4cm}
	{ \Huge \bfseries Informe Practica VI \\[0.4cm] }
\hrule \vfill
\end{centering}

\noindent
\centerline{Victor Tortolero \& Oswaldo Capriles} \newline

\vspace{8cm}
\centerline{Universidad de Carabobo}
\centerline{Facultad de Ciencia y Tecnologia}
\centerline{Arquitectura del Computador}
\centerline{\today}

\newpage
% %Fin Portada


\flushleft
\setlength{\parindent}{20pt}

%%CUERPO PRINCIPAl%%%%%%%%%%%%%%%%%%%%%%%%%%%%%%%%%%
\begin{centering} \section{Las Interrupciones} \end{centering}
	Una interrupción es una señal que recibe el procesador, ante la cual debe detener cualquier proceso que este realizando y darle prioridad al proceso o tarea que debe realizar según la interrupción que haya ocurrido. Las interrupciones también puede decirse que son un evento, que es accionado por una señal de hardware o software. 
	
	Las interrupciones de hardware son usadas para indicar al sistema operativo que cierto dispositivo requiere atención. Y fueron hechas con la finalidad de evitar código innecesario que solo esperaba por una entrada o señal de algún hardware.
	
	Cuando hablamos de las interrupciones por software, podemos hablar de las excepciones, que se usan para que el programa no reaccione de manera inesperada y sepa manejar los distintos errores que podrían causarse en tiempo de ejecución. Por ejemplo, en el caso de que el usuario ingrese una cadena y el programa espere un entero, en ese caso el programa no terminaría con un error, sino que haría algo al respecto, como notificarle al usuario y seguiría con la ejecución normal.Las interrupciones son comúnmente usadas para programas multitarea.
\newpage
	
	
\begin{centering} \section{Programacion Concurrente} \end{centering}
	Se habla de concurrencia cuando ocurren varios sucesos de manera contemporánea.
	En base a esto, la concurrencia en computación esta asociada a la ejecución de varios procesos que coexisten temporalmente.
	
	Para definir correctamente la programación concurrente, es necesario diferenciar programa de proceso.
	Un programa es un conjunto de sentencias o instrucciones que se ejecutan secuencialmente, y un proceso es básicamente un programa en ejecución.
	
	La concurrencia aparece cuando dos o mas procesos son contemporáneos. Un caso particular es el paralelismo o programación paralela. Los procesos pueden competir o colaborar entre si por los recursos del sistema. Por tanto, existen tareas de colaboración y sincronización
	
	La programación concurrente se encarga del estudio de las nociones de ejecución concurrente, asi como sus problemas de comunicación y sincronización.

\vspace{1cm}
\begin{centering} \section{Ejecución concurrente de programas y el temporizador} \end{centering}
	En programación concurrente un temporizador es un objeto que puede notificar a un proceso si ha transcurrido un cierto intervalo de tiempo o se ha alcanzado una hora determinada, los temporizadores permiten crear periodos temporales que una vez finalizados pueden generar señales que se envíen a otros procesos o al sistema operativo, informando sobre la finalización del temporizador. De igual forma también se pueden utilizar como contadores de cuenta atrás. cada temporizador esta asociado a un reloj.
%%FIN CUERPO PRINCIPAl%%%%%%%%%%%%%%%%%%%%%%%%%%%%%%%%%%


\newpage
\begin{centering} \section{Código Fuente} \end{centering}

\begin{lstlisting}[caption = Codigo Practica 5]
;Practica 6, Victor Tortolero, Oswaldo Capriles
DEFSEG	INICIO1,ABSOLUTE
SEG	INICIO1
JMP	INICIO

			ORG 	1BH
SIT1:	PUSH 	ACC
			CALL 	boing
			MOV 	R7, #01H
			POP 	ACC
			RETI

ORG 	100H
;Etiquetas de reloj
CERO	EQU 	30H
NUEVE	EQU 	39H
HOR1	EQU		20H
HOR2	EQU		21H
MIN		EQU		24H
SEG		EQU		27H
VAR 	EQU 	50H
;Etiquetas de carita
FACE 	EQU 	02H
BLAN 	EQU 	20H
FLAG	EQU 	F0

;Subprograma de la Carita
boing:	CJNE 	R1, #3FH, swag
			CPL		FLAG
swag:	CJNE 	R1, #38H, trans
			CPL		FLAG
trans:	JB 		FLAG, izq
der:	INC 	R3
			JMP 	step
izq:	DEC 	R3
step:	MOV		@R1, #BLAN
			MOV 	R1, 03H
			MOV 	@R1, #FACE
			RET
;Fin de Subprograma de la Carita

;Proceso del reloj %%%%%%%%%%%%%%%%%%%%%%%%%%%%%%%%%%%%%%%%%%%%%%%%%%%
RELOJ:	MOV		R7, #00H
				MOV 	A, P0
				MOV 	R0, #SEG
				CJNE 	A, #'a', ELSE1
				CALL 	IRELOJ
				RET
ELSE1:	CJNE 	A, #'A', ELSE2
				CALL 	IRELOJ
				RET
ELSE2:	CJNE 	A, #'d', ELSE3
				CALL 	DRELOJ
				RET
ELSE3:	CJNE 	A, #'D', ELSE4
				CALL 	DRELOJ
ELSE4: 	RET

IRELOJ:	CJNE 	@R0, #'9', ELSE5	;Revisando segundos
				MOV 	@R0, #'0'
				DEC 	R0
				CJNE 	@R0, #'5', ELSE5
				MOV 	@R0, #'0'
				MOV 	R0, #MIN			;Revisando minutos
				CJNE 	@R0, #'9', ELSE5
				MOV 	@R0, #'0'
				DEC 	R0
				CJNE 	@R0, #'5', ELSE5
				MOV 	@R0, #'0'
				MOV 	R0, #HOR1			;Revisando horas
				CJNE 	@R0, #'2', ELSE6
				INC 	R0
				CJNE 	@R0, #'3', ELSE5
				MOV 	@R0, #'0'
				DEC 	R0
				MOV 	@R0, #'0'
				RET
ELSE5:	INC 	@R0
				RET
ELSE6:	INC 	R0					;Continuacion de la revision de horas
				CJNE 	@R0, #'9', ELSE5
				MOV 	@R0, #'0'
				DEC 	R0
				INC 	@R0
				RET

DRELOJ:	CJNE 	@R0, #'0', ELSE7	;Revisando segundos
				MOV 	@R0, #'9'
				DEC 	R0
				CJNE 	@R0, #'0', ELSE7
				MOV 	@R0, #'5'
				MOV 	R0, #MIN			;Revisando minutos
				CJNE 	@R0, #'0', ELSE7
				MOV 	@R0, #'9'
				DEC 	R0
				CJNE 	@R0, #'0', ELSE7
				MOV 	@R0, #'5'
				MOV 	R0, #HOR1			;Revisando horas
				CJNE 	@R0, #'0', ELSE8
				INC 	R0
				CJNE 	@R0, #'0', ELSE7
				MOV 	@R0, #'3'
				DEC 	R0
				MOV 	@R0, #'2'
				RET
ELSE7:	DEC  	@R0
				RET
ELSE8:	INC 	R0					;Continuacion de la revision de horas
				CJNE 	@R0, #'0', ELSE7
				MOV 	@R0, #'9'
				DEC 	R0
				DEC 	@R0
				RET
;Fin del Proceso del reloj %%%%%%%%%%%%%%%%%%%

;Inicializando el Reloj en 00:00:00
INIT:	MOV 	20H, #'0'
			MOV 	21H, #'0'
			MOV 	22H, #':'
			MOV 	23H, #'0'
			MOV 	24H, #'0'
			MOV 	25H, #':'
			MOV 	26H, #'0'
			MOV 	27H, #'0'
			MOV 	R1, #38H
			MOV		@R1, #BLAN
			INC 	R1
			CJNE 	R1, #3FH, init2
			MOV 	@R1, #FACE
			MOV 	R3, 01H
			RET
INICIO:	MOV		PSW, #00H		
				CALL 	INIT
				MOV 	TMOD, #00100000B 	;Poniendo Timer 0 y Timer 1 en modo 2
				MOV 	R7, #00H
				SETB 	EA					;Activando Interrupciones
				SETB 	ET1					;Activando interrupcion del Timer 1
				SETB 	TR1					;Activando el Timer 1
				SETB 	PT1					;Dandole prioridad al Timer 1

LOOP: 	CJNE 	R7, #01H, LOOP
				CALL 	RELOJ
				JMP 	LOOP

EXIT:  	END
\end{lstlisting}
\newpage
\begin{thebibliography}{4}
	\bibitem{} \url{https://es.wikipedia.org/wiki/Computaci\%C3\%B3n\_concurrente}
	\bibitem{} \url{http://www2.ulpgc.es/hege/almacen/download/20/20233/tema1.pdf}
	\bibitem{} \url{https://books.google.co.ve/books?id=8nTnCgAAQBAJ\&pg=PT24\&lpg=PT24\&dq=Ejecuci\%C3\%B3n+concurrente+de+programas+y+el+temporizador\&source=bl\&ots=D27DHnY\_fK\&sig=t4YS5WZh7EwYU1s0eiRsCzVG3DE\&hl=en\&sa=X\&redir\_esc=y\#v=onepage\&q=Ejecuci\%C3\%B3n\%20concurrente\%20de\%20programas\%20y\%20el\%20temporizador\&f=false}
	\bibitem{} \url{https://en.wikipedia.org/wiki/Interrupt}
\end{thebibliography}
	
\end{document}